\documentclass[a4paper]{article}
\usepackage[utf8]{inputenc}
\usepackage{graphicx} % Required for including images
\usepackage[font=small,labelfont=bf]{caption}
\usepackage{courier}
\usepackage{amsmath}
\usepackage{color}
\usepackage{listings}
\usepackage{subcaption}

\usepackage{array}
\newcolumntype{M}[1]{>{\centering\arraybackslash}m{#1}}

\lstset{
    basicstyle=\ttfamily,
    language=Octave,
    morecomment = [l][\itshape\color{blue}]{\%}
}

\addtolength{\oddsidemargin}{-.875in}
\addtolength{\evensidemargin}{-.875in}
\addtolength{\textwidth}{1.75in}

\addtolength{\topmargin}{-.875in}
\addtolength{\textheight}{1.75in}

\setlength{\parindent}{0pt}
\setlength{\parskip}{0.5em}

\renewcommand{\figurename}{Figura}
\renewcommand{\tablename}{Tabla}

\newcommand{\bold}[1]{\textbf{\texttt{#1}}}

\title{TP 2: Algoritmos de Clasificación Supervisada}
\author{Giuliano Scaglioni}
\date{Septiembre 2019}

\begin{document}

\clearpage\maketitle
\thispagestyle{empty}

\newpage

\setcounter{page}{1}

\section{Deportes en el río}
  \subsection{Implementación}
  La implementación del programa se realizó en Go utilizando el algoritmo ID3 para construir el árbol de decision.

  \subsubsection{Entidades}
    En la implementación se definieron las siguientes entidades
    \begin{itemize}
      \item \bold{Example}: representa un ejemplo, su implementación es un mapa donde cada entrada tiene como clave el nombre del atributo y como valor, el valor que tiene ese atributo en el ejemplo.
      \item \bold{DecisionTree}: representa un árbol de decisión, para crear uno, se pasa una lista de ejemplos y el nombre del atributo a predecir. La implementación interna de esta entidad utiliza el algoritmo ID3 para crearlo. En la creación de un árbol de decisión, a su vez, se utilizan otras entidades, estas son:
      \begin{itemize}
        \item \bold{Node}: representa un nodo del arbol de decisión y define comportamiento para obtener el tipo de nodo, obtener la lista de hijos de este, agregar hijos, entre otros métodos necesarios para su representación.
        \item \bold{AttrNode}: es una implementación de Node para los nodos que representan un atributo.
        \item \bold{ValNode}: es una implementación de Node para los nodos que representan los valores posibles de un atributo.
        \item \bold{ClassNode}: es una implementación de Node para los nodos que representan una clase. Esta implementación no permite agregar hijos pues este tipo de nodos siempre es hoja.
      \end{itemize}
    \end{itemize}

    
\end{document}
